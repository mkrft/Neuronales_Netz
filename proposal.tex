\documentclass[12pt]{extarticle}
\usepackage[utf8]{inputenc}
\usepackage{cite}

\title{Trainieren eines neuronalen Netzes zur Optimierung von Rennstrategien im Motorsport am Beispiel der Formel 1}
\author{Tim Heckenberger, TIT19 \\ Manuel Kreft, TIT19 \\ Alexander Müller, TIT19}
\date{Oktober 2021}

\begin{document}
\setlength\parindent{0pt}

\maketitle
Ziel dieser Arbeit ist es ein neuronales Netz zu trainieren, welches in jeder Runde des Rennens entscheidet, ob ein Reifenwechsel vorteilhaft wäre und abhängig davon
die Entscheidung zum Boxenstopp trifft.\\
Somit soll die Arbeit von Rennstrategen im Umfeld des Motorsports erleichtert werden, in dem der Rennverlauf vor Beginn des Rennens durch das Wählen geeigneter Parameter simuliert und
durch die KI eine geeignete Strategie abgeleitet werden kann.\\
Die für das Training des neuronalen Netzes erforderlichen Daten sollen durch die Simulation eines Rennens erhoben werden. Der dafür erforderliche Simulator soll ebenfalls
im Rahmen dieser Arbeit erstellt werden und auf selbstgewählten Idealisierungen und Modellen basieren.\\
Durch diesen Umstand wird die Nähe zur Realität abhängig von den genutzten Annahmen, welche im möglichen zeitlichen Rahmen zunehmend erweitert werden können.
\end{document}