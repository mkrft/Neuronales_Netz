\section{Einleitung} %Alex

Nach dem Moor'schem Gesetz wird sich die Rechenleistung von Computern in regelmäßigen Zeiträumen verdoppeln. Dieses Gesetz konnte sich seit der kommerziellen Nutzung von Computern bewahrheiten, auch wenn hierzu immer wieder neue Technologiesprünge getan werden mussten. Mit dieser immer weiter zunehmenden Rechenleistung, welcher der Menschheit zur Verfügung steht, erweitert sich auch der Rahmen, in welchem diese Leistung Anwendung finden kann. So ist besonders der Nutzen im Bereich der Automatisierung von Prozessen in den letzten Jahrzehnten umfangreich angewachsen. Besondere Ansätze, welche diesen Prozess vorangetrieben haben, sind die zunehmende Verbreitung und Anwendung von künstlicher Intelligenz, welche besser in der Lage ist, komplexe Inhalte zu interpretieren und zu verwerten. Dies erlaubt ganze Prozesse vollständig und automatisch abzubilden, was zuvor historisch durch den limitierenden Faktor der Leistung undenkbar gewesen war. Somit sind Rechner heute schon in der Lage, Bilder zu Interpretieren und Sprache zu verstehen.
\\
\\
Um die gewünschte Wirkung einer künstlichen Intelligenz (KI) aber zu erreichen, muss diese wie im menschlichen Vorbild zunächst trainiert werden. Hierbei erheben sich einige Probleme, welche auch mit moderner Rechenleistung nur schwer abbildbar sind. So muss beispielsweise zunächst eine Datengrundlage geschaffen werden, welche die zu lösende Aufgabe umfangreich beschreibt, sodass ein Lernprozess auf Basis dieser Grundlage überhaupt den gewünschten Effekt erzielen kann und die KI somit in der Lage ist, einen Mehrwert zu leisten. Aus diesem Umstand heraus bildet sich das Problem, wo und wie man diese Daten beschaffen und deklarieren kann, damit sie dem Zweck dienlich sind. Dieser Prozess kann sehr langwierig sein und besonders Speicher und Rechenleistung stark kompromittieren. Ein anderer Ansatz zur Lösung dieses Problems wäre dagegen das Nachstellen der Wirklichkeit in einem simulierten Umfeld, in welchem die KI ohne Auswirkungen auf die Umwelt die erforderte Aufgabe selbst erlernen kann. Der Lernerfolg hängt dabei maßgeblich mit der Implementation der Simulation zusammen. Entsprechen die gewählten Modelle des simulierten Problems nicht ausreichend den Verhältnissen im echten Einsatzgebiet ist, kein Lernerfolg im Kontext des realen Problems zu erwarten.
\\
\\
Ziel dieser Arbeit ist somit die Lösung eines reellen Problems durch das Trainieren einer KI in einem simulierten Umfeld. Ein Beispiel für ein solches Problem ist durch den Motorsport beziehungsweise präziser durch die Formel 1 gegeben. In dieser müssen diverse Einflüsse bezüglich des Fahrzeuges verarbeitet werden, um eine Entscheidung bezüglich des Boxenstopps zu fällen. So muss im Grunde in jeder Runde entschieden werden, ob das Fahrzeug einen Boxenstopp durchführen soll. Sollte sich für den Boxenstopp entschieden werden, muss zusätzlich die gewünschte Reifenmischung ausgewählt werden. All diese Entscheidungen verfolgen dabei das Ziel, die eigene Rennzeit zu minimieren und somit einen Vorteil gegenüber den Konkurrierenden zu erreichen. Diese Aufgabe und Entscheidungsfindung soll im Rahmen dieser Arbeit durch eine KI abstrahiert werden. Hierzu soll zunächst ein Simulator geschaffen werden, welcher das Problem in gewünschter Auflösung darstellt und somit als Grundlage des Trainings dienen kann. Die Auflösung wird hierbei im Rahmen des Projekts entschieden, soll aber skalierbar gewählt werden, um somit die Komplexität nicht unnötig, zu erhöhen aber dabei auf die Notwendigkeit der Realitätsnähe zu achten. Als Umfeld wurde sich hierbei für die Formel 1 entschieden, aufgrund einer umfassenden Datenlage, welche es erlaubt, realitätsnahe Modelle zu generieren.