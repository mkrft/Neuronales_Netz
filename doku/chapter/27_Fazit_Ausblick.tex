\section{Fazit \& Ausblick}
% Grundlegend was erreicht wurde und welche Einschränkungen gelten
Das Ziel der Arbeit bestand in der Ermittlung einer optimalen Rennstrategie in einem simulierten Formel 1-Umfeld unter Verwendung der Möglichkeiten der künstlichen Intelligenz. Hierbei sollten durch ein KI-Modell auf Basis des Renngeschehens der optimale Zeitpunkt für einen Boxenstopp sowie die Art des Reifenwechsels bestimmt werden.\\\\
Zu diesem Zweck wurden auf Grundlage echter Renndaten Modelle entwickelt, welche verschiedenste rennstrategische Einflüsse abstrahieren, formalisieren und für die Verwendung in einer eigens entwickelten Simulationsumgebung berechenbar machen. So wurden unter Verwendung von polynomialer Regression Modelle entwickelt, welche den Einflussfaktor der Reifen hinsichtlich Rundengeschwindigkeit und Reifenverschleiß repräsentieren. Weiterhin wurden formale Abstraktionen für individuelle Faktoren wie Fahrzeug- und Fahrerleistung geschaffen. Zusätzlich erfolgte eine formale, berechenbare Abstraktion der Wechselwirkungen der Rennteilnehmenden in Form von Positionskämpfen und Überholmanövern.\\
Auf Basis der erstellten Modelle erfolgte die Implementierung eines Simulators, welcher ein vollständiges Formel 1-Rennen in Form diskreter Schritte simuliert und als Trainingsumfeld für das entwickelte KI-Modell dient.\\
Unter Verwendung der Simulationsumgebung erfolgte die Erarbeitung eines zur Ermittlung der optimalen Rennstrategie geeigneten KI-Modells. Hierbei hat es sich als zweckmäßig erwiesen, aufgrund des Charakters des Systems, welches kein überwachtes Lernen ermöglicht, das Prinzip des Reinforcement Learnings zu implementieren. Hierbei entscheidet sich das KI-Modell pro diskretem Simulationsschritt auf Grundlage von aktuellen Rennzustandsparametern für oder gegen einen Boxenstopp und wirkt durch diese Aktion auf die zukünftigen Zustände des Simulators ein und erhält entsprechend eine zustandsbezogene Belohnung. Als konkretes Lernverfahren wurde dabei der Deep-Q-Learning-Algorithmus verwendet.\\
Bei der Betrachtung der Ergebnisse, die ein trainiertes KI-Modell innerhalb der simulierten Rennumgebung liefert, zeigt sich, dass das hierdurch eine valide und potentiell optimale Rennstrategie ermittelt wird, die innerhalb der abstrahierten Rennumgebung gute Resultate liefert.
Hierbei ist jedoch festzuhalten, dass die entsprechende Rennstrategie aufgrund der Abstraktion des Simulators nur bedingte Schlüsse auf die Anwendbarkeit der ermittelten Strategie in der Realität zu lässt.
\\
% KI Entwicklung
Die Entwicklung der autonomen Lernkomponenten hat trotz der geringen Komplexität des Modells viel Zeit in Anspruch genommen. Dies resultiert aus der Instabilität des DQN-Algorithmus und der dadurch präsenten Empfindlichkeit gegenüber Parameteränderungen. Dennoch konnte eine zufriedenstellende Performance in der finalen Iteration des KI-Modells erreicht werden.
\\
Zusammenfassend konnte durch die Arbeit eine umfangreiche Grundlage für die weiterführende Entwicklung einer KI zur Entscheidung der Rennstrategie in der Formel 1 geschaffen werden. Um diese Grundlage entsprechend zu nutzen und die aktuellen Einschränkungen zu beheben, sollte für die weitere Entwicklung der Simulator umfassender gestaltet werden. Somit sollten vor allem weitere Faktoren untersucht und in die Betrachtung aufgenommen werden. Dazu zählen Reifen- und Streckentemperaturen, die Benzinlast der Fahrzeuge sowie Unfälle, Beschädigungen und daraus resultierende Safety-Car Phasen. Weiterhin sollte auch das Wetter und somit die Betrachtung der ausgesparten Regenreifen in den Umfang des Simulators aufgenommen werden. All diese Aspekte werden dem Simulator die nötige Komplexität geben, sodass mehrere Strategien und Vorgehensweisen möglich sind und die KI somit an Kompetenz in ihrer Rolle als Rennstratege gewinnt. Hierfür muss dann folglich die Komplexität der KI und die Möglichkeiten der KI erhöht und verbessert werden. So sollte die KI beispielsweise auch pro Runde entscheiden können, ob das entsprechende Fahrzeug langsamer oder mit Vollgas fährt. Diese Unterscheidung würde entsprechenden Einfluss auf die Reifen aber auch auf den Benzinverbrauch haben und somit die Strategie vollständig abbildbar machen. Entsprechend müsste die KI zwei zusätzliche Entscheidungen pro Runde treffen.\\
Zusätzlich bietet es sich an, eine Anbindung der KI an Live-Daten eines Formel 1 Rennens zu schaffen, um während des reellen Rennens die KI zu testen und zu untersuchen, ob die KI entsprechend die gleichen Entscheidungen fällt wie die echten Teams an der Rennstrecke. Hierbei sollte der Mensch zwar durch seine Kreativität weiterhin die Oberhand gegenüber der KI behalten, dennoch dürfte die KI im Mittel an die menschlichen Leistungen heranreichen. Dieser Umstand beschreibt die Hürde des Anwendungsspektrums und der Grenzen der Möglichkeiten von künstlichen Intelligenzen im Allgemeinen.
