%%%%%%%%%%%%%%%%%%%%%%%% Eigene Kommandos definieren %%%%%%%%%%%%%%%%%%%%%%%%
% Definition von \gqq{#1: text}: Text in Anführungszeichen
\newcommand{\gqq}[1]{\glqq #1\grqq}

% Definition von \footref{#1: label}
% Verweis auf bereits existierende Fußnoten mittels
\providecommand*{\footref}[1]{
	\begingroup
		\unrestored@protected@xdef\@thefnmark{\ref{#1}}
	\endgroup
\@footnotemark}

% Definition von \mypageref{#1: label}
% Kombination aus \ref{#1: label} und \pageref{#1: label}
\newcommand{\mypageref}[1]{\ref{#1} auf Seite \pageref{#1}}

% Definition von \myboxquote{#1: text}
% grau hinterlegte Quotation-Umgebung (für Zitate)
\newcommand{\myboxquote}[1]{
	\begin{quotation}
		\colorbox{boxgray}{\parbox{0.78\textwidth}{#1}}
	\end{quotation}
	\vspace*{1mm}
}

\makeatletter
\zref@newprop*{appsec}{}
\zref@addprop{main}{appsec}

% Definition von \applabel{#1: label}{#2: text}
% von \appsec{#1: text}{#2: label} zur Erzeugung des Labels verwendet)
\def\applabel#1#2{%
	\zref@setcurrent{appsec}{#2}%   
	\zref@wrapper@immediate{\zref@label{#1}}%
}

% Definition von \appref{#1: label}
% anstelle \ref{#1: label} zum referenzieren von Anhängen verwenden)
\def\appref#1{%
	\hyperref[#1]{\zref@extract{#1}{appsec}}%
}
\makeatother

% Definition von \appsection{#1: text}{#2: label}
% Ersetzt \section{#1: text} und \label{#2: label} für Anhänge)
\newcommand{\theappsection}[1]{Anhang \Alph{section}:~\protect #1}
\newcommand{\appsection}[2]{
	\addtocounter{section}{1}
	\phantomsection
	\addcontentsline{toc}{section}{\theappsection{#1}}
	\markboth{\theappsection{#1}}{}

	\section*{\theappsection{#1}}
	\applabel{#2}{Anhang \Alph{section}}
	\label{#2}
}

%%%%%%%%%%%%%%%%%% Für Bilder (#2 ist das Label) %%%%%%%%%%%%%%%%
%Definition \bild{#1: Groesse}{#2: Dateiname}{#3: Bildunterschrift}
\newcommand{\bild}[3]{
\begin{figure}[H]
    \centering
    \caption{#3} \label{fig:#2}
    \includegraphics[width=#1\textwidth]{Studienarbeit_F1/images/#2}
\end{figure}
}

%%%%%%%%%%%%%% Für Bilder mit Zitat (#2 ist das Label) %%%%%%%%%%
% Definition \bildzitat{#1 Groesse}{#2 Dateiname}{#3 Bildunterschrift}{#4 Seite Zitat}{#5 Zitatkey}
%%\begin{figure}[H]
    %\centering
    %\caption[#3]{#3\protect\footnotemark} \label{fig:my_label}
    %\includegraphics[width=#1\textwidth]{Studienarbeit_F1/images/#2}
%\end{figure}
%\footnotetext{\citet{#5}{#4}
%}


%%%%%%%%%%% Zitieren %%%%%%%%%%
% Neues Kommando Zitieren \zitat{#1: Seite}{#2: ID der Quelle}
\newcommand{\zitat}[2]{
\ifthenelse{\equal{#1}{}}
    {\footnote{Vgl. \citet{#2}}}
    {\footnote{Vgl. \citet{#2}{, S. #1}}}
}


\newcommand{\direktzitat}[2]{\ifthenelse{\equal{#1}{}}{\footnote{\citet{#2}}}{\footnote{\citet{#2}{,S. #1}}}}
